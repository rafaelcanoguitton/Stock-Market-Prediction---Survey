\documentclass[conference]{IEEEtran}
\ifCLASSOPTIONcompsoc
  \usepackage[nocompress]{cite}
\else
  \usepackage{cite}
\fi
\ifCLASSINFOpdf
\else
\fi

\begin{document}
\title{Stock Market prediction using \\Artificial Intelligence}
\author{\IEEEauthorblockN{Rafael Isaac Cano Guitton}
\IEEEauthorblockA{School of Computer Science\\
Universidad Católica San Pablo \\
Arequipa, Perú \\
Email: rafael.cano@ucsp.edu.pe}
}
\maketitle
\begin{abstract}
Abstract not required for this presentation.
\end{abstract}
\IEEEpeerreviewmaketitle
\section{Introduction}
The stock market is a place which is a collection of markets where stocks of a firm are traded (bought and sold)\cite{M2018}.
Its behaviour is considered chaotic \cite{Singh2016} and it has always been ambiguous for investors
because of several influential factors. This unreliable behaviour make investors live the potential to lose big sums of money, or to treat
the stock market as gambling. There's a correlation between investment psychology and market behaviour. With a predictive model, we can give a 
solution for impulsive market selling and buying. Assuring a percentage of reliability in trading this impulsive action can be reduced. While there's many automated ways to assist investors' decisions
in a timely manner\cite{nabipour2020predicting}.
\\\\
The main reason behind prediction is buying stocks that are likely to increase in price, and then selling stocks that are probably going to fall\cite{nabipour2020predicting}. 
Stock prices react to events related to businees performances or overseas markets. Investors judge on the basis of technical analysis, such as company's charts\cite{Akita2016}. 
The purpose is to analyze current state of the art techniques for stock market accuracy\cite{Singh2016}.
\\\\
Now it is difficult to predict market trends and many Artificial Intelligence (AI) approaches
have been investigated to predict them automatically. For example, investment simulation analysis with artificial
 markets\cite{Akita2016}. The stock market's instability and nonlinearity cause problems for data analysts to develop a predictive model \cite{nabipour2020predicting}.
\\\\
We'll focus on \textit{Machine Learning} and Deep Learning models,
\\\\
The ample research literature, combined with the vast underlying models, tasks and training methods make it very
hard to identify the most appropriate approach or the most effective.
\\\\
And there's where our challenge resides. On this survey we aim to compare techniques based on Artificial Intelligence such as Deep Learning and Machine Learning.

% %\hfill mds
% %\hfill August 26, 2015
% %\subsection{Subsection Heading Here}
% %Subsection text here.
% %\subsubsection{Subsubsection Heading Here}
% Subsubsection text here.
% \section{Conclusion}
% The conclusion goes here.
\section{Conclusion}
It can be concluded that in this survey will review results of Deep learning and Machine learning models, 
well compare them according to results that show lower risks.
\\\\
We need to first clasify all models according to it's approach so that we can extract the best one from every
category, next we'll compare the best exponent. Comparing it's advantages and disadvantages will help us deduce
the lowest risk model for every case.
\\\\
For comparison purposes we'll compare prediction values for certain stocks with real stock market behavior.
\\\\
\ifCLASSOPTIONcompsoc
  \section*{Acknowledgments}
\else
  \section*{Acknowledgment}
\fi
Not required for this presentation.

\bibliographystyle{IEEEtran}
\bibliography{references}
\end{document}